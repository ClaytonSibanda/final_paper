\documentclass[plain]{sigplanconf}
\usepackage{balance} % For balanced columns on the last page
\usepackage{amsmath}
\usepackage[T1]{fontenc}
\usepackage{lmodern}
\usepackage{graphicx}
\usepackage{amssymb}
\usepackage{tikz}
\usepackage{array}
\usepackage{longtable}
\usepackage{subcaption}
\usepackage[
bookmarksopen,
bookmarksdepth=2,
breaklinks=true
]{hyperref}
\usepackage{natbib}
\setcitestyle{square,sort,comma,numbers}
\makeatletter
\def\BState{\State\hskip-\ALG@thistlm}
\makeatother

\makeatletter
\def\@copyrightspace{\relax}
\makeatother
\begin{document}
	\title{Community Networks QoS Monitoring System}

	\authorinfo{Clayton Sibanda}
	{Department of Computer Science\linebreak University of Cape Town\linebreak South Africa}
	{}
	\maketitle

	\begin{abstract}
	\paragraph{}
	Characterising the internet through measurements has become very important to both end users and network
	managers. Most end users want to know what is causing their network based applications to slow down or take too long to respond. On the other hand network managers want to troubleshoot and detect faulty nodes in the network. The rise of cyber attacks and abuse of networks also makes it of paramount importance for network managers to continuously keep track of what is happening in their network.
	\paragraph{}
	In this paper we presented the design and creation of a visualiser for a quality of service monitoring platform for community networks. Such a visualiser is important in that it can help in monitoring network activity and identifying anomalous behaviour. The visualiser will include graphed and textual data from TCP, ping, traceroute and DNS measurements.
	\paragraph{}
	We evaluated the visualiser's ability and effectiveness in communicating huge amounts of network data both to technical and non-technical users. A user centred design was adopted to design the visualiser and usability tests were conducted to assess the usability and accuracy of the visualiser.
	\paragraph{}
	
	\end{abstract}
	\begin{CCSXML}
		<ccs2012>
		<concept>
		<concept_id>10003033.10003079.10011704</concept_id>
		<concept_desc>Networks~Network measurement</concept_desc>
		<concept_significance>100</concept_significance>
		</concept>
		</ccs2012>
	\end{CCSXML}
	\ccsdesc[100]{Networks~Network measurement}
	\keywords
	Community Networks, Quality of Service
	
\section{Background}
\paragraph{}
An network/internet measurement platform is an infrastructure of dedicated probes that periodically run network measurement tests on the internet/network\cite{7076582}.

\paragraph{}
 Network measurement platforms from the perspective of end systems normally play an important role both for researchers who want to develop general insight into how the Internet functions, and for general practitioners who want to diagnose individual performance issues \cite{Dhawan:2012:FBN:2398776.2398786}. Use cases of these measurement platforms vary widely based on the type of a user and the end goal of the measurement \cite{Ford:2018:RWR:3243157.3243167}. One example of a use case is how in  previous studies these platforms have been employed to understand the overall network topology of the internet\cite{7076582}.
\paragraph{}
For managers and regulators measurement platforms can provide a different view of the internet e.g routing, topology, addressing and naming, security, dataplane performance or impairment and traffic matrices\cite{Ford:2018:RWR:3243157.3243167}.
\paragraph{}
In some cases these platforms are used to detect anomalous behaviour in the network which could be caused by users abusing the network or general cyber attacks. Internet service providers(ISP) on the other hand use monitoring platforms to evaluate the Quality of Service(QoS) experienced by their users\cite{7076582}. Consumers can also use such measurements to confirm whether the ISP is living up to its Service-Level Agreement(SLA) offers \cite{7076582}.
\paragraph{}
In general measurement platforms are said to provide the engineering to bridge the gap between practice and research in terms of coordination to allow a measurement to scale  and representation producing simple and useful results \cite{Ford:2018:RWR:3243157.3243167}.
\subsection{Quality Of Service}
Quality of Service(QoS) is concerned about the network
delivery capacity and resource availability to users\cite{5430142}. In
other words one would say QoS is about fast internet
access for the user and low latency. 
\paragraph{}
However there are many non-uniform views about QoS by different
stakeholders. Some say QoS refers to the ability of the
network to offer packet transfer in a faster way \cite{5430142}.
\paragraph{}
At the same time other organisations have maintained that
QoS has to do with the degree of conformance to user-specified needs \cite{article}. These two different views raise questions of how network level QoS measurements and control relate to the user perception of a service \cite{5430142}.
\paragraph{}
The two main QoS parameters are network latency
and delivery speed(bandwidth) \cite{5430142}. As a result QoS is
considered poor if any of them is affected from their normal position i.e if bandwidth is low or if latency is high.
\subsection{Community Networks}
Community networks are IP-based networks that are built, operated and owned by communities of citizens. These networks are sometimes ran by non-profit organisations working together with local stakeholders in the community. 

Other scholars define them as large scale, self-organized and decentralized networks, built and operated by citizens for citizens \cite{Braem:2013:CRC:2500098.2500108}.
\paragraph{}
These networks are normally built as a mixture of both wireless and wired links. This calls for different routing and systems and protocols to be employed in these networks. In some cases there many wireless links with a limited number of wired links. This makes them predominantly wireless networks\cite{Braem:2013:CRC:2500098.2500108}.
\paragraph{}
Providing connectivity to community networks can be a challenging task since nodes use diverse access technologies and display a great deal of mobility \cite{Plagemann2008}.
Due to their non-standard architecture and the use of diverse technology it has proven to be difficult to deploy conventional network measurement platforms on community networks.


\section{Introduction}
	\paragraph{}
Characterising the internet through measurements has become very important to both end users and network
managers. Most end users want to know what is causing their network based applications to slow down or take too long to respond. On the other hand network managers want to troubleshoot and detect faulty nodes in a network. The rise of cyber attacks and abuse of networks also makes it of paramount importance for network managers to continuously keep track of what is happening in their network.
\paragraph{}
The goal of this is to present the design and creation of a visualiser for a quality of service monitoring platform for community networks. Such a visualiser is important in that it can help in monitoring network activity and identifying anomalous behaviour.
\paragraph{}
The platform is built as case study for the iNethi community network at ocean view. There are two main types of target users for this platform. The first ones are non-technical end-users of the network. These are people who live in the community and use the network for different purposes. The second group are technical(advanced) end-users with a working knowledge of networking protocols. These would include networking researchers from tertiary institutions and network managers.
\paragraph{}
For non-technical users, the goal of the platform is to enable them to get various statistics calculated based on their past network usage. These statistics will also include websites or services which consumes most of their data and the services they visit the most.
\paragraph{}
On the other hand, the goal for technical users, is to allow them run different measurements on the network with intention to monitor network activities. The results of the measurements are to be displayed on a rich web based visualiser that employs interactive graphs to present each type of measurement. These measurements are to be run on a number of mobile probes deployed haphazardly on the network. Fig 1 shows how the whole system is to be connected together.

\begin{figure*}
	\centering
	\includegraphics[width=0.7\linewidth]{images/system}
	\caption{Image of the system showing how the visualiser was connected to everything}
	\label{fig:system}
\end{figure*}
\input{sections/background}
\section{Literature Review}
\subsection{Prominent network measurement tools}
There are a number of network measurement software
that are currently deployed to monitor and observe patterns in different networks.
\paragraph{}
The first one that we discuss is PerfSONAR(Performance focused
Service Oriented Network monitoring ARchitecture)\cite{10.1007/11596141_19}. This is a service infrastructure that is web based and is used to collect and publish network performance monitoring\cite{article2}. Its main goal is to make it easy to handle end-to-end performance problems on paths going through several networks \cite{article2}. It does this by using a set of services to deliver performance measurements to an agreed upon network environment.

\paragraph{}
LiveLab is a method used to measure smartphone usage in the field and to measure wireless networks through smartphone users\cite{article3}. This platform employs a tool that is custom built for Iphones only to enable researchers to track users in the field. LiveLab also provides a comprehensive in-device logging of smartphone usage and measurement of wireless
networks\cite{article3}.The tool takes advantage of the mobility
of users and the ability of smartphones to switch connections between multiple routers and that way researchers are able to gain large amounts of data\cite{article3}.

\paragraph{}
RIPE ATLAS is also another tool used extensively
to measure network performance. It uses thousands of
distributed probes and anchors as measurement devices.
The tool can perform IPv4 and IPv6 traceroute, ping,
DNS, NTP\cite{7076582}.
\paragraph{}
PlanetLab is a measurement platform used for testing of new network services. However PlanetLab is rather unusable due to unpredictable load issues and tendency of nodes to be located in a national research network\cite{7076582}.
PeerMetric is a measurement tool used to measure P2P network performance experienced by broadband hosts\cite{7076582}.
\paragraph{}
Mobiperf is an android based application that is used to collect mobile network measurements. On the backend it uses a data collection server to collect and aggregate data. The app periodically checks in with the measurement server which sends it a list of measurement tasks to perform. These measurement tasks include ping, traceroute, HTTP GET, DNS lookup, TCP Throughput, IPv4/v6 compatibility check and UDP Burst. Each task is given a respective set of measurement parameters.

\section{Design and Implementation}
\subsection{Approach}
The purpose of a network visualiser is to allow users to analyse and get insight into how they use the network\cite{Ruan2018}. It also allows network managers(advanced users) and researchers to schedule and perform measurements. The results of the measurements are also displayed on interactive graphs for further analysis.
\paragraph{}
For this project an iterative user-centred design(UCD) approach was adopted. This entails that the user is involved throughout the whole design process so that the visualisation produced meets every need that the user has\cite{Andrews:2006:EIV:1168149.1168151}. The goal of this approach is to find out the needs and tasks of the user and then design based on that\cite{Dylggduu}. This visualisation platform is developed for a system that is to be deployed at a local community network called iNethi in Cape Town. Users of this visualisation platform are expected to be network researchers, network managers and general users of the network.
\paragraph{}
This approach consisted of three phases:early envisioning phase, the global specification phase and the detailed specification phase\cite{Kulykinbook}.
\paragraph{}
The early envisioning phase entails the analysis of the users, the environment they are in and their tasks. This will enable us to profile users and gather requirements in the process\cite{Kulykinbook}. In the context of a network visualiser, the would mean understanding the type of data that user would to see visualised and analyse. A number of methods can be employed to obtain this which include surveys, interviews and focus groups\cite{Kulykinbook} \cite{Abras04user-centereddesign}.
\paragraph{}
In the global specification phase and the detailed phase, a designer comes up with a solution and presents it to users\cite{Abras04user-centereddesign} \cite{Kulykinbook}. Each phase may contain multiple iterations of design and analysis with evaluations taking place in all phases\cite{Abras04user-centereddesign}.
\paragraph{}
\begin{figure}[b]
	\centering
	\includegraphics[width=0.7\linewidth]{images/img1}
	\caption{User-centered visualisation design process\cite{Abras04user-centereddesign}}
	\label{fig:img1}
\end{figure}

\paragraph{}
Users generally have different abilities and tasks to perform on the visualisation tool. Due to these differences it is imperative that a detailed analysis of the users, their environment and their tasks is performed before the design of the visualisation platform.

\subsection{Early envisioning}
The first iteration of the design process was early envisioning. At this iteration the main goal was to understand potential users(nature of the users) and have a clear understanding of the tasks they perform and the environment in which they perform the tasks in\cite{Valiati:2006:TTG:1168149.1168169}. The information gathered here is used to command the design of the visualisation and its functionality.

\subsubsection{Requirements Gathering}
To gather the needed information, the team attended a meeting where issues concerning the iNethi network were to be discussed. The meeting also consisted of Ocean View community members, University Of Cape Town(UCT) networking researchers,UCT professors and other technical staff. Some of the stakeholders in the meeting were members of the UCT Information and Communications Technologies for Development(ICT4D). In this meeting the use cases of the project were outlined to potential users and various stakeholders.
\paragraph{}
Specific questions concerning the network and relevance of the project were answered by stakeholders. The team also gained understanding of the nature of potential users and the environment in which the visualisation tool will be used. Some potential users also suggested features for the visualisation that would be useful to any other users. The feedback obtained ensured that the visualisation design would meet the needs of potential users and stakeholders.

\subsubsection{Implementation}
After the meeting with stakeholders and potential users, the team set out to develop a prototype of the visualisation. An interactive prototype was created using adobe xd. The prototype was limited in functionality but allowed the user to navigate from one page to another while performing different operations.The prototype did not use any real world data, the goal was mainly to focus on the user interface design.
\paragraph{}
Ben Shneiderman's golden rules of design were adopted and closely followed while designing the prototype. To strive for consistency, a blue and white theme was adopted and used in all the pages of the application. Visuals were also used to reduce short-term memory while using the application. The image on figure 2 shows the prototype screens and how the user would navigate from one to the next.


\begin{figure*}
	\centering
	\includegraphics[width=1\linewidth]{images/proto}
	\caption{First prototype of the visualizer}
	\label{fig:proto}
\end{figure*}

\subsubsection{Evaluation}
After weeks of designing the prototype, four potential users were invited from the community of Ocean View to UCT. Amongst them were two community leaders that are involved in the running of the iNethi network. The other two were members of the community that would use the platform as end users.
\paragraph{}
A usability study which involved semi-structured interviews was carried out with potential users. Users were allowed to go through the app and encouraged to be talking and pointing out what they are doing at each stage. While this was happening the team was observing users, their facial reaction and what they are saying. At beginning of each session the user will be presented with the first page of the app and they were encouraged to navigate through the app from there on and perform different operations. To avoid bias, users did the testing of the application separately.
\paragraph{}
At the end of each session the users were asked to provide feedback in written form. Results indicated that three out of four users found the application not difficult to use and the same number said that the application would be useful to the community. On top of the that some users suggested new features which they thought would be very useful if incorporated into the application.

\subsection{Global Specification}
In this phase, solutions that have been developed are proposed and presented to users and stakeholders\cite{Kulykinbook}.
\subsubsection{Analysis and Requirements Gathering}
The feedback from the first prototype was gathered and put together for the next phase of the design approach. This feedback was to be used to command the initial design of the actual application.
\subsubsection{Implementation}
Based on the feedback gathered from the previous phase, it was decided that the development of the actual user interface be started. The user interface was designed to be a responsive web based system that the user can use comfortably on their mobile devices.
\paragraph{}
React, a JavaScript front end framework was chosen for the development of the web interface. With over two million developers using React, it is currently one of the most used frontend frameworks in web development\cite{githubreact}. It is also developed by Facebook which makes it a very reliable and stable framework with a lot of support for developers\cite{githubreact}. As a result the use of React ensured that the web interface was developed in the minimum amount of time. A chart library called ReChart was used to produce graphs to display measurement results.
\paragraph{}
The web interface was then connected to a Java based request handler. The request handler was used to handle writing and reading requests sent by users from the web interface. Image on Figure 3 shows how the whole system was connected together.
\begin{figure*}
	\centering
	\includegraphics[width=0.7\linewidth]{images/system}
	\caption{Image of the system showing how the visualiser was connected to everything}
	\label{fig:system}
\end{figure*}


\subsubsection{Evaluation}
A second usability study was carried out during this phase. This would allow users to evaluate an actual web interface that is interactive and deals with real world data.
\paragraph{}
Users we allowed to go through the visualiser following different instructions which involved saying what they were doing while going through it. They were also observed while performing different tasks. The team was particularly interested in their reactions, emotions and hurdles as they performed different tasks.
\paragraph{}
At the end of the study, users were interviewed about their experience with the visualiser. They were also given a Google form with a few questions to fill anonymously.

\subsection{Final Visualisation Design}
The final visualisation design was designed based on the feedback received from the two previous phases to ensure the satisfaction of users. On top of using Reactjs to ease development, CSS bootstrap was used to ease the styling of the web interface and to make it responsive. This ensured that user experience is consistent across all devices with highest priority given to mobile ones. 
\paragraph{}
The visualiser has two parts to it, first a part where users can schedule measurements(measurement initiator) and part where they can view results of the past measurements(data visualiser). Users can schedule any of the five measurements which are trace route, ping , DNS lookup, HTTP and TCP  Speed Test. 
\paragraph{}
The measurement part of the visualizer therefore has five modes. On changing to a new mode, the user interface gives feedback to the user by changing the title of the measurement type. This is consistent with Shneiderman's eight golden rules of design\cite{Shneiderman:2016:DUI:3033040}. Feedback is also given to the user by highlighting the measurement type selected on the side bar. 
\paragraph{}
The form contains six fields for different inputs. To adhere to the eight golden rules of design, all the forms have constraints on them to prevent errors. For example the date-time form can only take a date later than a specified date and also the job interval form can only take numbers\cite{Shneiderman:2016:DUI:3033040}.
There are  Figure 4. how this would look like to the user. 

\begin{figure}
	\begin{subfigure}{.5\textwidth}
		\centering
		\includegraphics[width=.8\linewidth]{images/http}
		\caption{HTTP mode}
		\label{fig:sfig1}
	\end{subfigure}
	\begin{subfigure}{.5\textwidth}
		\centering
		\includegraphics[width=.8\linewidth]{images/ping}
		\caption{ping mode}
		\label{fig:sfig2}
	\end{subfigure}
	\begin{subfigure}{.5\textwidth}
	\centering
	\includegraphics[width=.8\linewidth]{images/troute}
	\caption{traceroute mode}
	\label{fig:sfig3}
\end{subfigure}
	\begin{subfigure}{.5\textwidth}
	\centering
	\includegraphics[width=.8\linewidth]{images/dlookup}
	\caption{DNS lookup mode}
	\label{fig:sfig4}
\end{subfigure}
\begin{subfigure}{.5\textwidth}
	\centering
	\includegraphics[width=.8\linewidth]{images/tcp}
	\caption{TCP speed test mode}
	\label{fig:sfig5}
\end{subfigure}
	\caption{Different modes of the measurement page}
	\label{fig:fig}
\end{figure}

\paragraph{}
The second part of the visualizer the results page. Like the measurement page, it follows a multi modal design where a user can toggle between modes. This ensures consistency across the user interface\cite{goldenrules}. It also maintains the same colours schemes like the first part of the visualiser. A side bar to select a different result type was also added to this page. Highlighting a selected item is still done for feedback.
\paragraph{}
On selecting the result type, a user is presented with a detailed view of the measurement results. At the top of the list is a button that is used to toggle between graph view and detailed view of the results. On selecting the graph, the text on the button changes from "show detail" to "show graph" hence giving meaningful feedback to the user. Figure 5. shows how this was implemented. T












\section{Final evaluation and results}
\section{Limitations}
\subsection{Usability Tests}
\subsubsection{Participants}
Taking the approach of User-Center Design means that the right users should be sourced and involved in the design process. These users need to be easily reachable and available throughout all the design phases. This means that most of our users had to be community members at Ocean View and be users of the iNethi network. Sourcing a large number of participants with a networking domain knowledge was a challenging endeavour. Some participants were not comfortable with using computers navigating a website. As result, not all visualisation functionalities could be tested with them.
\paragraph{}
Since the platform also allows researchers to perform measurements on the network, some of our potential users were computer science students and academics. One of the requirements would be for them to be familiar with the iNethi network or any community network. At the end it proved to be a challenge to source students who had past experience with the iNethi network or any community network.
\paragraph{}
Having a small number of participants meant that our sample was not representative of all the users of the iNethi network.

\subsubsection{Environment}
Usability tests were conducted in the UCT ICT4D seminar room. Ocean View community members were invited over to the Computer Science department. This was a different environment from the one they normally use the network in. As a result it is suspected that some results were influenced by the environment they are not used to.
\subsection{Visualisation}
This limitation brings to light the difference between usability(as it refers to interfaces) and data usability. Usability is normally used to refer to how well the end-users are using the application\cite{Luciana}. However on visualisation interfaces, users get to interact with both interface widgets and data that is used for decision making\cite{Luciana}. Hence data usability is a more appropriate term for visualisation interfaces. According to \cite{Luciana}, data usability is used to describe the quality of data on visualisation interfaces
\paragraph{}
The are many factors that could affect the data, some of them include the way in which it is presented and noise during the collection of data. In this project, measurements for different networking protocols and tools were ran. Some of them, like TCP have a lot of parameters. Given that a graph cannot display all the parameters that a protocol has, some of them were discarded and not displayed. To counteract this loss, a detailed view of all the measurements was included.
\paragraph{}
The way in which the data is structured also affects the effectiveness of a visualiser. It does so by allowing end-users to make assumptions about the type and amount of data delivered. These assumptions can affect the decisions drawn from the visualiser by end-users\cite{Knight2001}. 
\paragraph{}
The data for this project is collected and written to InfluxDB, a time series database which is designed to handle real time data. The data is then written to MongoDB, a nosql database that allows the storing of unstructured data.The visualiser therefore displays data that is queried from MongoDB. The data is formatted before being sent to the web interface, and this has significant effect on how the data is presented to the user.



\section{conclusion}
In this paper, we presented the design of a quality of service monitoring platform for iNethi community network. For this design, a user centred design approach was adopted and used over two iterations(phases). The final visualisation was built as a result of the success of the previous iterations.
\paragraph{}
The visualiser consists of two parts. A measurement initiator and data visualizer. The measurement initiator allows users to schedule and run experiments on different probes while the data visualizer displays to the user the results of the measurements performed by the probes. 
\paragraph{}
On the measurement initiator, the user can schedule and run five different measurement types which are HTTP, DNS lookup, ping, traceroute and TCP speed test. The data visualizer employs interactive graphs and JSON texts to display measurement results.
\subsubsection{Future works}
Ideally the platform will need to deliver real time data to the visualiser. However this cannot be done with user's phones being used as probes. This is because the phones will have to continuously measure and send data to the server, and this may affect the user's phones. As a result, a viable probe for real-time data will be a device like a raspberry pi installed within the network.
\paragraph{}
Currently, we use email addresses for authorising users to access data and view network data. This happens to be an insecure way of authorising users. We therefore suggest the use of both an email and password to authenticate all users.
\subsection{ACKNOWLEDGEMENTS}
Thanks to project team members David Kheri and Meluleki Dube for their contributions and Dr Josiah Chavula for his guidance throughout the project as project supervisor. An additional thank you to Dr Maria Keet for her input as second reader.. 
	%\nocite{*}
	\bibliographystyle{acm}
	\bibliography{references}
%	\input{sections/appendix}
\end{document}