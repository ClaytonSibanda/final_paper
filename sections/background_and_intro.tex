\section{Background and Introduction}
\paragraph{}
An network/internet measurement platform is an infrastructure of dedicated probes that periodically run network measurement tests on the internet/network\cite{7076582}.

\paragraph{}
 Network measurement platforms from the perspective of end systems normally play an important role both for researchers who want to develop general insight into how the Internet functions, and for general users who want to diagnose individual performance issues \cite{Dhawan:2012:FBN:2398776.2398786}. Use cases of these measurement platforms vary widely based on the type of a user and the end goal of the measurement \cite{Ford:2018:RWR:3243157.3243167}. One example of a use case is how in  previous studies these platforms have been employed to understand the overall network topology of the internet\cite{7076582}.
\paragraph{}
For managers and regulators measurement platforms can provide a different view of the internet e.g routing, topology, addressing and naming, security, dataplane performance or impairment and traffic matrices\cite{Ford:2018:RWR:3243157.3243167}.
\paragraph{}
In some cases these platforms are used to detect anomalous behaviour in the network which could be caused by users abusing the network or general cyber attacks. Internet service providers(ISP) on the other hand use monitoring platforms to evaluate the Quality of Service(QoS) experienced by their users\cite{7076582}. Consumers can also use such measurements to confirm whether the ISP is living up to its Service-Level Agreement(SLA) offers.

\subsection{Quality Of Service}
Quality of Service(QoS) is concerned about the network
delivery capacity and resource availability to users. In
other words one would say QoS is about fast internet
access for the user and low latency. 
\paragraph{}
However there are many non-uniform views about QoS by different
stakeholders. Some say QoS refers to the ability of the
network to offer packet transfer in a faster way \cite{5430142}.
\paragraph{}
At the same time other organisations have maintain that
QoS has to do with service quality for the user. These
two different views raise questions of how network level QoS measurements and control relate to the user perception of a service \cite{5430142}.
\paragraph{}
The two main QoS parameters are network latency
and delivery speed(bandwidth) \cite{5430142}. As a result QoS is
considered poor if any of them is affected from their normal position i.e if bandwidth is low or if latency is high.

