\section{conclusion}
In this paper, we presented the design of a quality of service monitoring platform for iNethi community network. For this design, a user centred design approach was adopted and used over two iterations(phases). The final visualisation was built as a result of the success of the previous iterations.
\paragraph{}
The visualiser consists of two parts. A measurement initiator and data visualizer. The measurement initiator allows users to schedule and run experiments on different probes while the data visualizer displays to the user the results of the measurements performed by the probes. 
\paragraph{}
On the measurement initiator, the user can schedule and run five different measurement types which are HTTP, DNS lookup, ping, traceroute and TCP speed test. The data visualizer employs interactive graphs and JSON texts to display measurement results.
\subsubsection{Future works}
Ideally the platform will need to deliver real time data to the visualiser. However this cannot be done with user's phones being used as probes. This is because the phones will have to continuously measure and send data to the server, and this may affect the user's phones. As a result, a viable probe for real-time data will be a device like a raspberry pi installed within the network.
\paragraph{}
Currently, we use email addresses for authorising users to access data and view network data. This happens to be an insecure way of authorising users. We therefore suggest the use of both an email and password to authenticate all users.
\subsection{ACKNOWLEDGEMENTS}
Thanks to project team members David Kheri and Meluleki Dube for their contributions and Dr Josiah Chavula for his guidance throughout the project as project supervisor. An additional thank you to Dr Maria Keet for her input as second reader.. 